\documentclass[12pt,a4paper,twocolumn,twoside]{article}
\usepackage[a4paper,total={20cm, 28.5cm},headsep=0.3cm]{geometry}
\usepackage[utf8]{inputenc}
\usepackage{sectsty}
\usepackage{titlesec}
\usepackage{amsfonts}
\usepackage{enumitem}
\usepackage{amsmath}

\titleformat{\subsection}{\small\bfseries}{\thesection}{1em}{}
\titleformat{\subsubsection}{\footnotesize\bfseries}{\thesection}{1em}{}
\sectionfont{\fontsize{10}{8}\selectfont}

\titlespacing*{\section}{0pt}{0.2cm}{0.1cm}
\titlespacing*{\subsection}{0pt}{0.2cm}{0.1cm}
\titlespacing*{\subsubsection}{0pt}{0.2cm}{0.1cm}

\setlength{\columnseprule}{1pt}


\def\abs#1{\lvert #1 \rvert}
\def\real{\mathbb{R}}
\def\suminfty#1#2{\sum_{n=#1}^\infty #2}
\def\der#1#2#3{\frac{d #1}{d #2}\left( #3 \right)}
\def\parder#1#2#3{\frac{\partial #1}{\partial #2}\left( #3 \right)}
\def\forvar#1#2#3{E(#1)=#2 \quad Var(#1)=#3}
\def\rref{\stackrel{G-J}{\longrightarrow}}
\def\vector#1{\bold{\stackrel{\rightarrow}{#1}}}

\begin{document}
%
%Subspace
%
\section*{Subspace}
Et subspace av $\real^n$ er et subset V av $\real^n$, som oppfyller:
\begin{enumerate}[topsep=0pt,itemsep=0pt, partopsep=0pt]
    \item Nonemptiness: $\vector{0}\in V$
    \item Closure under additiion: $\vector{u},\vector{k}\in V \Leftrightarrow \vector{u}+\vector{k}\in V$
    \item Closure under multiplication: $\vector{u}\in V \Leftrightarrow c\vector{u}\in V$
\end{enumerate}
Alle subspaces er et span og alle span er et subspace
\subsection*{Finne ut om et subset er et subspace}
\begin{itemize}[topsep=0pt,itemsep=0pt, partopsep=0pt]
    \item Er subsettet et span? Kan det skrives som et span?
    \item Kan det bli skrevet som et columnspace til en matrise?
    \item Kan det bli skrevet som nullspacet til en matrise
    \item Er det hele $\real^n$ eller $\{\vector{0}\}$
    \item Kan det skrives som en type subspace?
    \begin{itemize}[topsep=0pt,itemsep=0pt, partopsep=0pt]
        \item Eigenspace
        \item Ortogonal complement etc\dots
    \end{itemize}
    \item Kan en bekrefte de tre kravene til et subspaces er oppfylt?
\end{itemize}
\section*{Basis:}
La V være et subspace av $\real^n$. En basis til V vil da være et sett av
vektorer $\{\vector{v_1},\vector{v_2},\dots\vector{v_n}\}$ slik at:
\begin{enumerate}[topsep=0pt,itemsep=0pt, partopsep=0pt]
    \item $V = Span\{\vector{v_1},\vector{v_2},\dots\vector{v_n}\}$
    \item $\{\vector{v_1},\vector{v_2},\dots\vector{v_n}\}$ Er lineært uavhengig.
\end{enumerate}
\section*{Rank Theorem:}
\begin{itemize}[topsep=0pt,itemsep=0pt, partopsep=0pt]
    \item $Rank(A) = Dim(Col(A)) = Dim(Row(A))$
    \item $Nullity(A) = Dim(Null(A))$
\end{itemize}
Hvis $A$ er en $m \times n$ matrise, Da:\\ $Rank(A)+Nullity(A)=n$
\section*{Invertibel matrise teorem:}
La $A$ være en kvadratisk matrise. Følgende utsagn er ekvivalent:
\begin{enumerate}[topsep=0pt,itemsep=0pt, partopsep=0pt]
    \item $A$ er invertibel
    \item Redusert trappeform til $A$ er identitetsmatrisen
    \item $A\vector{x}=0$ Har ingen løsninger annet en den trivielle
    \item $Nul(A) = \{\vector{0}\}\quad Nullity(A) = 0$
    \item Kolonnene til $A$ er lineært uavhengig
    \item Kolonnene til $A$ former en basis for $\real^n$
    \item $Col(A) = \real^n$
    \item $Rank(A)=n$
    \item $A\vector{x}=b$ er konsistent for alle $b$ i $\real^n$
    \item $A\vector{x}=b$ har en unik løsning for alle b i $\real^n$
    \item $det(A)\neq 0$
    \item $\vector{0}$ er ikke en egenvektor til $A$
\end{enumerate}
\section*{Determinant:}
\subsection*{Definisjon:}
Determinanten er en funksjon:\\
$det:\{n\times n matrise\}\rightarrow \real$\\
Som oppfyller følgende atributter:
\begin{enumerate}[topsep=0pt,itemsep=0pt, partopsep=0pt]
    \item Å legge til en multippel av en rad til en annen rad, endrer ikke determinanten
    \item Skalere en av radene til $A$ med en skalar $c$ multipliserer determinanten med $c$
    \item Bytte to rader av en matrise, multipliserer determinanten med $-1$
    \item Determinanten til identitetsmatrisen er 1
\end{enumerate}
\subsection*{Atributter til determinanten:}
\begin{itemize}[topsep=0pt,itemsep=0pt, partopsep=0pt]
    \item Hvis $A$ har en 0 kollone eller 0 rad så er determinanten 0
    \item Hvis $A$ er triangulær, så er determinanten produktet av elementene langs diagonalen
    \item $det(A^{-1}=\frac{1}{\det(A)})$
    \item $det(AB) = det(A)det(B)$
    \item $det(A^T)=det(A)$
    \item Hvis en matrise $A$ har to like rader, så er $det(A)=0$
    \item Determinanten er volumet til paralellepipeden spent ut av kolonnene til en matrise
\end{itemize}
\section*{Egenvektor og egenverdier:}
La $A$ være en $n\times n$ matrise:
\begin{enumerate}[topsep=0pt,itemsep=0pt, partopsep=0pt]
    \item En egenvektor av $A$ er en ikkenull vektor $\vector{v}$ i $\real^n$
            slik at $A\vector{v}=\lambda \vector{v}$
    \item En egenverdi av A er en skalar $\lambda$, slik at likningen $A\vector{v}=\lambda\vector{v}$, har en ikketriviell løsning.
\end{enumerate}
Obs! Egenvektor er ved definisjon ikkenull, men egenverdier kan være 0
\subsection*{Det karakteristiske polynomet:}
\subsubsection*{Definisjon:}
Karakteristiske polynomet til en matrise $A$ er funksjonen:\\
$f(\lambda)=det(A-\lambda I_n)$
\subsubsection*{Theorem:}
La A være en kvadratisk matrise, og la $f(\lambda)=det(A-\lambda I_n)$ være dens karakteristiske polynom.
Da er $\lambda$ en egenverdi til A hviss $f(\lambda) = 0$

\section*{Similære matriser:}
\subsection*{Definisjon:}
To kvadratiske matriser $A$ og $B$ er similære, hvis det finnes en ivertibel matrise $C$ slik at $A=CBC^{-1}$
\subsection*{Atributter til similære matriser:}
\begin{itemize}[topsep=0pt,itemsep=0pt, partopsep=0pt]
    \item Refleksivitet: A er similær med seg selv
    \item Symmetri: A er similær med B $\Leftrightarrow$ B er similær med A
    \item Transitivitet: A er similær med B, og B er similær med C, da er A similær med C
    \item $A=CBC^{-1}\rightarrow A^n=CB^nC^{-1}$
    \item $A=CBC^{-1}$: $\vector{v_1}$ er en egenvektor til A $\Rightarrow$ $C^{-1}\vector{v_1}$ er en egenvektor til B
    \item $A=CBC^{-1}$: $\vector{v_2}$ er en egenvektor til B $\Rightarrow$ $C\vector{v_2}$ er en egenvektor til A
\end{itemize}

\section*{Diagonalisering:}
\subsection*{Definisjon:}
En kvadratisk matrise A er diagonaliserbar hvis den er similær til en diagonal matrise.
\subsection*{Theorem:}
En kvadratisk matrise A er diagonaliserbar hviss A har n lineært uavhengige egenvektorer.
I tilfellet, $A=CDC^{-1}$ så gjelder:\\
$C=[v_1,v_2,\dots,v_n] \; D=\begin{bmatrix}
    \lambda_1 & 0 & \dots & 0\\
    0 & \lambda_2 & \dots & 0\\
    \vdots & \vdots & \ddots & 0\\
    0 & 0 & 0 & \lambda_n
\end{bmatrix}$\\
Hvor $v_1,v_2\dots v_n$ og $\lambda_1,\lambda_2, \dots,\lambda_n$ er egenvektorene/verdiene til A

\section*{Ortogonal kompliment:}
La $V$ være et underrom i $\real^n$, da er det ortogonale komplimentet til $V$, alle vektorer som står ortogonalt på $V$\\
$V^\perp=\{\vec{x}|x\cdot\vec{y} = 0, y\in V\}$\\
OBS! Må ikke forveksles med ortogonal basis!

\section*{Ortogonal basis:}
En ortogonal basis, er en basis $\{\vec{v_1},\vec{v_2},\dots,\vec{v_n}\}$ for et subspace $V$, der $\vec{v_i}\cdot\vec{v_j}=0$.
Altså alle vektorene i basisen, er ortogonale. 

\subsection*{Ortonormal basis:}
I en ortonormal basis er alle vektorene av lengde 1.

\subsection*{Gram-Schmidt:}
Brukes for å finne ortogonal basis til et underrom:\\
Gitt en basis $\{\vec{v_1},\vec{v_2},\dots,\vec{v_m}\}$ for et underrom $V \subseteq \real^n$. Er en ortogonal basis vektorene $\{\vec{u_1},\vec{u_2},\dots,\vec{u_n}\}$ hvor:
\begin{enumerate}[topsep=0pt,itemsep=0pt, partopsep=0pt]
    \item $\vec{u_1}=\vec{v_1}$
    \item $\vec{u_2}=\vec{v_2}- Proj_{\vec{u_1}}\vec{v_2}$
    \item $\vec{u_3}=\vec{v_3}- Proj_{\vec{u_1}}\vec{v_3}-Proj_{\vec{u_2}}\vec{v_3}$
    \item $\vec{u_m}=\vec{v_m}- \sum_{i=1}^m Proj_{\vec{u_m}}\vec{v_m}$
\end{enumerate}

\section*{Ortogonal diagonalisering:}
En matrise A er ortogonal diagonaliserbar hviss:
\begin{itemize}[topsep=0pt,itemsep=0pt, partopsep=0pt]
    \item A er symmetrisk om diagonalen.
    \item $A=PDP^{-1}$, hvor P er en ortogonal matrise.
    \begin{itemize}
        \item En ortogonal matrise er en matrise, hvor kolonnene er ortogonale enhetsvektorer
        \item $Q^T=Q^{-1}\Leftrightarrow Q^TQ=I$
    \end{itemize}
\end{itemize}

\section*{Komplekse Matriser:}
\begin{itemize}[topsep=0pt,itemsep=0pt, partopsep=0pt]
    \item En kompleks matrise A er Hermitsk hviss $A=A^T$
    \item En komplex matrise A er unitær hviss $A\cdot A=I$
    \begin{itemize}[topsep=0pt,itemsep=0pt, partopsep=0pt]
        \item Kolonnene til A er en ortonormal basis til $\mathbb{C}$ 
    \end{itemize}
\end{itemize}
\end{document}